\setcounter{chapter}{1}
\setcounter{section}{0}

\begin{center}
	\vspace{0.5cm}
	\huge{\bf 第1章}
	\par
	\vspace{1cm}
	\hrulefill
	\par
	\vspace{1cm}
	\huge{\bf 序論}
	\par
	\vspace{0.5cm}
	\hrulefill
	\vspace{1cm}
	\par
	
	\begin{flushleft}
		\large{{\bf 本章では,本研究の背景と目的,および本論文の構成について述べる.}}
	\end{flushleft}
\end{center}

\addcontentsline{toc}{chapter}{\protect\numberline{第1章}{序論}}

\newpage

\section{研究背景}
パソコンやスマートフォンなどのデジタルデバイスは,現代社会において情報アクセスやコミュニケーションに不可欠なツールとなっている.
教育,就労,行政手続き,医療,娯楽など,あらゆる生活場面においてデジタル技術の活用が前提となりつつあり,デジタルデバイスを操作できるか否かが社会参加の機会を大きく左右する時代となった.
しかし,すべての人がデジタルデバイスを容易に操作できるわけではない.
特に,身体機能に制限がある人々にとって,従来の入力デバイスであるキーボードの操作には困難が伴う.
例えば,脳性麻痺,本態性振戦,パーキンソン病などの神経疾患や,加齢に伴う運動機能の低下により,手の震えや不随意運動を持つ人々にとって,キーボードによる正確な文字入力は困難であり,デジタル社会への参加を阻む障壁となっている.

このような入力の困難に対応するため,運動機能障害者向けの入力支援技術として,音声入力,視線入力,特殊キーボードなど様々なアプローチが開発されてきた.
しかし,これらの技術にはそれぞれ課題がある.
また,カメラベースの手認識技術も研究が進んでいるが,既存研究の多くは健常者の安定した動作を前提としており,震えや不随意運動を伴う入力への対応は十分に検討されていない.

\section{研究目的}
本研究の目的は,一般的なWebカメラのみを用いて,運動機能障害者の手指の動きから入力意図を推定するキーボード入力支援システムを開発することである.
特にキーボード入力に焦点を当てているのは,プログラミング,データ入力,文書作成などの専門的な作業において,キーボードが依然として最も効率的な入力手段だからである.
このシステムにより,特殊な機器の追加なしに既存のキーボードをそのまま活用できる,低コストな入力支援が実現できる.
本システムの実現により,運動機能障害者がキーボードを活用できるようになり,教育や就労の機会拡大,さらに社会参加の促進につながることが期待される.

\section{本論文の構成}
本論文の構成は以下の通りである.
第2章では,運動機能障害者向けの入力支援技術,カメラベースの手認識技術,震え補正と運動意図推定に関する既存研究を概観し,本研究の位置づけを明確にする.
第3章では,提案システムの設計について述べる.
4点キャリブレーションによる座標系の統一,MediaPipeを用いた手指軌跡の取得,18次元動的特徴量の設計,GRUモデルによる入力意図推定について説明する.
第4章では,評価実験の設計について述べる.モデル比較のため,オフライン評価と研究者本人による予備的なリアルタイム評価の両方で複数のモデルを比較し,最適なモデルを選定する.また,選定したモデルを用いた被験者実験によるシステムの実用性評価の設計についても述べる.
第5章では,被験者実験による評価実験の結果を示し,考察を行う.
第6章では,本研究の結論と今後の展望を述べる.