\newpage
\setcounter{chapter}{5}
\setcounter{section}{0}

\begin{center}
	\vspace{0.5cm}
	\huge{\bf 第5章}
	\par
	\vspace{1cm}
	\hrulefill
	\par
	\vspace{1cm}
	\huge{\bf 考察}
	\par
	\vspace{0.5cm}
	\hrulefill
	\vspace{1cm}
	\par
	
	\begin{flushleft}
		\large{{\bf 本章では,評価実験の結果に基づき考察を行う.}}
	\end{flushleft}
\end{center}

\addcontentsline{toc}{chapter}{\protect\numberline{第5章}{考察}}

\newpage

\section{モデル選択の妥当性}

モデル比較実験において,GRUがオフライン評価・リアルタイム評価ともに最高精度を達成した.
GRUはLSTMと同等の長期依存関係学習能力を持ちながら,パラメータ数が約24\%少なく学習・推論ともに高速であるため,リアルタイム処理に適している.
一方,CNNとTCNはオフライン評価では93\%以上の精度を達成したが,リアルタイム評価では大幅に性能が低下した.
これは,畳み込み層による局所的な特徴抽出が,可変長の時系列パターン全体を捉える上で不十分であったためと考えられる.
この結果から,指先軌跡からの入力意図推定には,時系列の長期依存関係を学習できるRNN系モデルが適していることが示唆される.

\section{特徴量重要度分析}

図\ref{fig:importance}に,Permutation Importanceによる特徴量重要度分析の結果を示す.
分析の結果,指先のx座標(finger\_x)とx方向速度(vel\_x)が最も高い重要度を示した.
これは,キーボード入力において左右方向の移動と速度情報が特に重要であることを示している.
次いで,y方向速度(vel\_y)と最近傍キーへの相対座標(rel\_key1\_x, rel\_key1\_y, rel\_key2\_y)が比較的高い重要度を示した.
これに対し,震えの特性を表す振幅(amplitude\_x, amplitude\_y)と方向転換頻度(direction\_change)はほとんど寄与しておらず,0に近い値あるいは負の値を示した.

この結果は,手指の「位置」と「速度」が入力意図推定の主要な情報であり,「加速度」や「揺れの振幅」といった高次の動的特徴は本実験では必ずしも必要ではないことを示唆している.
ただし,これは健常者による学習データに基づく分析である.
健常者の入力では震えがほとんど発生しないため,震え関連の特徴量が有効に機能しなかった可能性がある.
実際の運動機能障害者のデータでは,不随意運動の特性を捉えるために振幅や方向変化といった特徴が重要になる可能性も考えられる.

\begin{figure}[htb]
\centering
\includegraphics[width=\columnwidth]{../figures/importance.png}
\caption{特徴量重要度分析}
\label{fig:importance}
\end{figure}

\section{震え対応の有効性}

振戦模擬入力の結果は,提案システムの震え対応能力を示すものである.
振戦模擬入力条件では,96.5\%の試行で意図しないキーが物理的に押下された.
このような状況下でも,システムは65.0\%の試行で正しい入力意図を推定できた.
これは,震えを除去せず生の軌跡から直接学習するアプローチの有効性を示唆している.
また,Top-3精度は90.0\%を維持しており,「自動補正」と「候補提示」の2段構えの救済機能が,運動機能障害者の入力効率を大幅に改善する可能性がある.

\section{オフライン評価とリアルタイム評価の差異}

GRUのオフライン評価(Top-1精度98.0\%)と実用性評価の通常入力(Top-1精度83.8\%)の間には約14ポイントの精度差が見られた.
この差異の原因として,実験参加者の多様性が考えられる.
オフライン評価およびモデル比較実験は研究者本人のデータで学習・評価したが,実用性評価は10名の異なる実験参加者で実施した.
入力スタイルや手の動きには個人差があり,未知の実験参加者に対する汎化性能が低下した可能性がある.
これらの課題に対しては,より多様な実験参加者からデータを収集してモデルを学習することで,汎化性能の向上が期待できる.

\section{実用性に関する考察}

通常入力でのTop-1精度83.8\%は,実用的な入力支援システムとしては改善の余地がある.
しかし,Top-3精度が97.6\%と高いことから,候補選択型インターフェースやハイブリッド方式などの使用形態が有効と考えられる.
また,震えを伴う入力においても64.9\%の精度で正しい意図を推定できることは,運動機能障害者のキーボード入力支援として一定の有効性を示している.
特に,誤入力時の救済率65.0\%は,震えにより誤ったキーを押してしまった場合でも,システムが自動的に修正候補を提示できることを意味しており,ユーザーの入力負担軽減に寄与すると考えられる.

\section{限界と制約}

本評価実験にはいくつかの限界がある.

\subsection{実験参加者に関する制約}

本実験は健常者のみを対象としており,実際の運動機能障害者での評価は行っていない.
振戦模擬入力は「押下直前に意図的に別のキーを押す」よう指示したものであり,実際の振戦による不随意運動の特性を完全に再現するものではない.
実際の振戦では,目標キーに向かう軌跡全体を通じて震えが生じるため,「途中までは正しい軌跡である」という本実験の前提が成り立たない可能性がある.
そのため,実際の運動機能障害者における性能は本実験の結果と異なる可能性がある.

\subsection{評価条件の制約}

入力タスクは37文字のランダム文字列のみであり,長文や自然言語文での評価は行っていない.
また,評価指標も認識精度に限定しており,入力速度や長時間使用における疲労度,主観的な使いやすさといった実用性に関わる観点からの評価は実施していない.

\subsection{システムの制約}

学習データが研究者1名から収集したものに限られており,多様な入力スタイルへの汎化性能が限定的である可能性がある.
加えて,現在のシステムは事前学習されたモデルを使用しており,個々のユーザーの入力特性に適応する機能は実装していない.

キャリブレーションについては,4点のキーをクリックする操作が必要であり,これ自体が運動機能障害者にとって困難である可能性がある.
また,本システムはJIS配列キーボードを前提としており,異なる配列のキーボードには対応していない.

さらに,本システムは2次元の座標(x, y)のみを使用しており,高さ方向(z軸)の震えには対応していない.
単一のカメラでは高さ情報の取得が困難であり,上下方向の不随意運動がある場合には別途対応が必要となる.

また,運動機能障害者は指の感覚が低下している場合があり,目視でキーを確認しながら入力する必要が生じるため,タッチタイピングが困難になる可能性も考えられる.
この場合,候補提示UIが視線移動を要求すると入力効率が低下する恐れがあり,UIの視認性向上が課題となる.
加えて,タイピング動作自体が腕や手,指先の疲労を引き起こしやすいことも想定されるため,少ない打鍵数での入力を可能にする機能の実装も今後の課題である.