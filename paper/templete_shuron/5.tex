\newpage
\setcounter{chapter}{5}
\setcounter{section}{0}

\begin{center}
	\vspace{0.5cm}
	\huge{\bf 第5章}
	\par
	\vspace{1cm}
	\hrulefill
	\par
	\vspace{1cm}
	\huge{\bf 結果と考察}
	\par
	\vspace{0.5cm}
	\hrulefill
	\vspace{1cm}
	\par
	
	\begin{flushleft}
		\large{{\bf 本章では,実用性評価実験の結果を示し,考察を行う.}}
	\end{flushleft}
\end{center}

\addcontentsline{toc}{chapter}{\protect\numberline{第5章}{結果と考察}}

\newpage

\section{実用性評価結果}

\subsection{入力条件別性能}

表\ref{tab:realtime_results}に被験者10名の平均性能を示す.
通常入力ではTop-1精度83.8\%,Top-3精度97.6\%を達成した.
振戦模擬入力ではTop-1精度64.9\%,Top-3精度90.0\%であった.
図\ref{fig:condition_comparison}に入力条件別の精度比較を示す.

\begin{table}[htb]
\caption{入力条件別性能評価(被験者10名の平均)}
\begin{center}
\begin{tabular}{|l|c|c|} \hline
入力条件 & Top-1精度 & Top-3精度 \\ \hline \hline
通常入力 & 83.8\% & 97.6\% \\ \hline
振戦模擬入力 & 64.9\% & 90.0\% \\ \hline
\end{tabular}
\end{center}
\label{tab:realtime_results}
\end{table}

\begin{figure}[htb]
\centering
\includegraphics[width=\columnwidth]{../figures/condition_comparison.png}
\caption{入力条件別精度比較}
\label{fig:condition_comparison}
\end{figure}

\subsection{振戦模擬入力における救済率}

振戦模擬入力条件では,被験者の操作により全試行の96.5\%で物理的な誤入力(隣接キー等の押下)が発生した.
このような状況下でも,システムはTop-1精度64.9\%を達成した.
これは実質的に,誤入力が発生した際にシステムが自動的に正しい意図を推定できた割合(救済率65.0\%)を示している.
図\ref{fig:rescue_rate_distribution}に救済率の分布を示す.
また,Top-3精度は90.0\%を維持しており,自動補正が外れた場合でも9割の確率で候補から正解を選択可能である.

\begin{figure}[htb]
\centering
\includegraphics[width=\columnwidth]{../figures/rescue_rate_distribution.png}
\caption{救済率の分布}
\label{fig:rescue_rate_distribution}
\end{figure}


\subsection{被験者別結果}

図\ref{fig:participant_comparison}に被験者別のTop-1精度を示す.
通常入力のTop-1精度は被験者間で64.9\%〜97.3\%の範囲であり,比較的安定した性能を示した.
振戦模擬入力では40.5\%〜81.1\%の範囲となり,被験者間のばらつきが大きくなった.
これは,振戦の模擬方法が被験者によって異なることに起因すると考えられる.

\begin{figure}[htb]
\centering
\includegraphics[width=\columnwidth]{../figures/participant_comparison.png}
\caption{被験者別Top-1精度}
\label{fig:participant_comparison}
\end{figure}


\subsection{軌跡の可視化}

図\ref{fig:trajectory_isomura}および図\ref{fig:trajectory_furuta}に,振戦模擬入力時の実際の指先軌跡の可視化例を示す.
図\ref{fig:trajectory_isomura}では,目標キーである「h」キーに向かって移動した後,押下直前に「g」キーを押している.
しかし軌跡の大部分は「h」キーに向かっているため,システムは正しく「h」キーと推定できている.
図\ref{fig:trajectory_furuta}も同様に,軌跡の全体的な動きから正しい入力意図を読み取れていることが確認できる.
このように,システムは軌跡全体のパターンから,押下直前の急激な動きが誤入力であることを判断できる.

\begin{figure}[htb]
\centering
\includegraphics[width=\columnwidth]{../figures/trajectory_isomura.png}
\caption{被験者Aの軌跡可視化例}
\label{fig:trajectory_isomura}
\end{figure}

\begin{figure}[htb]
\centering
\includegraphics[width=\columnwidth]{../figures/trajectory_furuta.png}
\caption{被験者Bの軌跡可視化例}
\label{fig:trajectory_furuta}
\end{figure}


\section{考察}

\subsection{モデル選択の妥当性}

モデル比較実験において,GRUがオフライン評価・リアルタイム評価ともに最高精度を達成した.
GRUはLSTMと同等の長期依存関係学習能力を持ちながら,パラメータ数が約24\%少なく学習・推論ともに高速であるため,リアルタイム処理に適している.
一方,CNNとTCNはオフライン評価では93\%以上の精度を達成したが,リアルタイム評価では大幅に性能が低下した.
これは,畳み込み層による局所的な特徴抽出が,可変長の時系列パターン全体を捉える上で不十分であったためと考えられる.
この結果から,指先軌跡からの入力意図推定には,時系列の長期依存関係を学習できるRNN系モデルが適していることが示唆される.

\subsection{特徴量重要度分析}

図\ref{fig:importance}に,Permutation Importanceによる特徴量重要度分析の結果を示す.
分析の結果,指先のx座標(finger\_x)とx方向速度(vel\_x)が最も高い重要度を示した.
これは,キーボード入力において左右方向の移動と速度情報が特に重要であることを示している.
次いで,y方向速度(vel\_y)と最近傍キーへの相対座標(rel\_key1\_x, rel\_key1\_y, rel\_key2\_y)が比較的高い重要度を示した.
これに対し,震えの特性を表す振幅(amplitude\_x, amplitude\_y)と方向転換頻度(direction\_change)はほとんど寄与しておらず,0に近い値を示した.

この結果は,手指の「位置」と「速度」が入力意図推定の主要な情報であり,「加速度」や「揺れの振幅」といった高次の動的特徴は必ずしも必要ではないことを示唆している.
ただし,これは健常者による学習データに基づく分析であり,実際の運動機能障害者のデータでは異なる傾向を示す可能性がある.
特に,不随意運動の特性を捉えるためには,振幅や方向変化といった特徴が重要になる可能性も考えられる.

\begin{figure}[htb]
\centering
\includegraphics[width=\columnwidth]{../figures/importance.png}
\caption{特徴量重要度分析}
\label{fig:importance}
\end{figure}

\subsection{震え対応の有効性}

振戦模擬入力の結果は,提案システムの震え対応能力を示すものである.
振戦模擬入力条件では,96.5\%の試行で意図しないキーが物理的に押下された.
このような状況下でも,システムは65.0\%の試行で正しい入力意図を推定できた.
これは,震えを除去せず生の軌跡から直接学習するアプローチの有効性を示唆している.
また,Top-3精度は90.0\%を維持しており,「自動補正」と「候補提示」の2段構えの救済機能が,運動機能障害者の入力効率を大幅に改善する可能性がある.

\subsection{オフライン評価とリアルタイム評価の差異}

GRUのオフライン評価(Top-1精度98.0\%)と実用性評価の通常入力(Top-1精度83.8\%)の間には約14ポイントの精度差が見られた.
この差異の原因として,被験者の多様性が考えられる.
オフライン評価およびモデル比較実験は研究者本人のデータで学習・評価したが,実用性評価は10名の異なる被験者で実施した.
入力スタイルや手の動きには個人差があり,未知の被験者に対する汎化性能が低下した可能性がある.
これらの課題に対しては,より多様な被験者からデータを収集してモデルを学習することで,汎化性能の向上が期待できる.

\subsection{実用性に関する考察}

通常入力でのTop-1精度83.8\%は,実用的な入力支援システムとしては改善の余地がある.
しかし,Top-3精度が97.6\%と高いことから,候補選択型インターフェースやハイブリッド方式などの使用形態が有効と考えられる.
また,震えを伴う入力においても64.9\%の精度で正しい意図を推定できることは,運動機能障害者のキーボード入力支援として一定の有効性を示している.
特に,誤入力時の救済率65.0\%は,震えにより誤ったキーを押してしまった場合でも,システムが自動的に修正候補を提示できることを意味しており,ユーザーの入力負担軽減に寄与すると考えられる.

\subsection{限界と制約}

本評価実験にはいくつかの限界がある.
まず,被験者に関する制約として,本実験は健常者のみを対象としており,実際の運動機能障害者での評価は行っていない.
振戦模擬入力は意図的に別のキーを押すよう指示したものであり,実際の振戦による不随意運動の特性を完全に再現するものではない.
そのため,実際の運動機能障害者における性能は本実験の結果と異なる可能性がある.

次に,評価条件の制約として,入力タスクは37文字のランダム文字列のみであり,長文や自然言語文での評価は行っていない.
また,評価指標も認識精度に限定しており,入力速度や長時間使用における疲労度,主観的な使いやすさといった実用性に関わる観点からの評価は実施していない.

さらに,システム自体の制約として,学習データが研究者1名から収集したものに限られており,多様な入力スタイルへの汎化性能が限定的である可能性がある.
加えて,現在のシステムは事前学習されたモデルを使用しており,個々のユーザーの入力特性に適応する機能は実装していない.