\newpage
\setcounter{chapter}{6}
\setcounter{section}{0}

\begin{center}
	\vspace{0.5cm}
	\huge{\bf 第6章}
	\par
	\vspace{1cm}
	\hrulefill
	\par
	\vspace{1cm}
	\huge{\bf 結論}
	\par
	\vspace{0.5cm}
	\hrulefill
	\vspace{1cm}
	\par
	
	\begin{flushleft}
		\large{{\bf 本章では,本研究の成果をまとめ,今後の展望について述べる.}}
	\end{flushleft}
\end{center}

\addcontentsline{toc}{chapter}{\protect\numberline{第6章}{結論}}

\newpage

\section{本研究のまとめ}

本研究では,運動機能障害者のキーボード入力を支援するため,一般的なWebカメラのみで動作する低コストな入力意図推定システムを提案した.
提案システムは,4点キャリブレーションによるキーボード座標系の設定,MediaPipeによる手指軌跡の取得,18次元動的特徴量の設計,GRUモデルによる可変長時系列学習という一連の処理により,不随意運動を伴う軌跡から入力意図を推定する.
18次元の動的特徴量は,指先の正規化座標,最近傍キーへの相対座標と距離,速度・加速度,および震えの振幅・方向転換頻度から構成され,震えを除去すべきノイズとして扱うのではなく,生の軌跡データから直接学習するアプローチを実現した.

モデル比較実験では,LSTM,CNN,GRU,TCNの4種類を比較し,GRUがオフライン評価において最も高い認識精度(Top-1精度98.0\%)を達成した.
GRUはパラメータ数も約167Kと最小であり,リアルタイム処理に適している.
被験者10名による実用性評価では,通常入力でTop-1精度83.8\%,Top-3精度97.6\%を達成した.
振戦模擬入力では96.5\%の試行で物理的な誤入力が発生したが,システムはTop-1精度64.9\%(救済率),Top-3精度90.0\%を達成し,誤入力からの救済機能の有効性を確認した.


\section{本研究の貢献}

\subsection{学術的貢献}

本研究の学術的貢献として,まず,震えの特性(振幅,方向転換頻度)を含む18次元の動的特徴量を設計し,運動機能障害者の入力意図推定に有効であることを示した点が挙げられる.
従来の研究では位置情報や速度情報が中心であったが,本研究では震えそのものを特徴量として取り込むことで,不随意運動を含む軌跡からも入力意図を推定できることを実証した.

次に,従来の震え除去アプローチとは異なり,震えを個人の運動特性として学習するアプローチの有効性を実証した点が挙げられる.
既存の震え補正技術はKalmanフィルタなどで震えを除去するアプローチが主流であったが,本研究ではフィルタリングによる情報損失を回避し,震えに含まれる動作特性も活用して入力意図を推定することが可能となった.

また,入力意図推定タスクにおいて,GRU,LSTM,CNN,TCNの4モデルを比較し,可変長の手指軌跡データにはGRUが最も適していることを示した.
特にCNNとTCNはオフライン評価では90\%以上の精度を達成したがリアルタイム評価では大幅に性能が低下しており,本タスクにおけるRNN系モデルの優位性を明らかにした.

\subsection{社会的貢献}

本研究の社会的貢献として,追加ハードウェアなしに,低コストで運動機能障害者のキーボード入力を支援できるシステムを実現した点が挙げられる.
既存の視線入力デバイスや特殊キーボードは高価であり,導入が困難な場合が多いが,本システムは一般的なWebカメラのみで動作するため,経済的な障壁を大幅に低減できる.
これにより,高価な支援技術を導入できない運動機能障害者に対しても,デジタルデバイスへのアクセス手段を提供する可能性を示した.


\section{今後の展望}

第5章で述べた限界と制約を踏まえ,今後の展望を述べる.

まず,被験者に関する課題として,本研究の学習データおよび評価実験は健常者のみを対象としている.
実際の運動機能障害者の手指軌跡は,不随意運動の種類や程度が個人によって大きく異なるため,今後,倫理審査を経た上で実際の運動機能障害者(本態性振戦,パーキンソン病患者など)を対象とした評価実験を実施し,システムの実用性を検証する必要がある.

次に,評価条件の拡充として,本研究では37文字のランダム文字列のみを入力タスクとしたが,実際の使用場面では長文や自然言語文の入力が想定される.
また,評価指標も認識精度に限定しており,入力速度や長時間使用における疲労度,主観的な使いやすさといった実用性に関わる観点からの評価も必要である.
今後はより多様なテキストを用いた評価実験を実施し,これらの複合的な指標で検証する予定である.

さらに,システムの改善として,学習データの拡充と個人適応機能の実装が挙げられる.
現状の学習データは研究者1名から収集したものに限られているため,より多くの被験者からデータを収集し,多様な入力スタイルに対応できるモデルを学習する.
加えて,個々のユーザーの障害特性に適応する個人適応機能を実装することで,さらなる精度向上が期待できる.

また,言語モデルとの統合も今後の課題である.
本研究ではランダム文字列を用いてシステム単体の性能を測定したが,言語モデルとの統合により,文脈情報を活用したより高精度な入力支援の実現を目指す.

これらの課題に取り組むことで,運動機能障害者のデジタルアクセシビリティが向上し,教育や就労の機会拡大,社会参加の促進につながることを期待する.