\documentclass[twocolumn, a4paper, 9pt]{jarticle}
% --------------- package ---------------
\usepackage[dvipdfmx]{graphicx}
\usepackage{yokou}
\usepackage{url}

%-----参考文献の文字サイズ変更------
\makeatletter
  \renewenvironment{thebibliography}[1]{%
    \global\let\presectionname\relax
    \global\let\postsectionname\relax
    \section*{\refname}\@mkboth{\refname}{\refname}%
    \list{\@biblabel{\@arabic\c@enumiv}}%
          {\settowidth\labelwidth{\@biblabel{#1}}%
          \leftmargin\labelwidth
          \advance\leftmargin\labelsep
          % 文字サイズ?
          \setlength\baselineskip{12pt}
          % アイテム間のマージン
          \setlength\itemsep{0.4zh}
          \@openbib@code
          \usecounter{enumiv}%
          \let\p@enumiv\@empty
          \renewcommand\theenumiv{\@arabic\c@enumiv}}%
    \sloppy
    \clubpenalty4000
    \@clubpenalty\clubpenalty
    \widowpenalty4000%
    \sfcode`\.\@m}
    {\def\@noitemerr
      {\@latex@warning{Empty `thebibliography' environment}}%
    \endlist}
\makeatother

% --------------- 本文 --------------- 

\title{日本語タイトル}
\etitle{English Title}
\author{00FMIXX 苗字 名前}{研究指導教員 教授 岩井 将行}
\eauthor{FIRST LAST}{MASAYUKI IWAI}

\begin{document}
\maketitle
% \maketitle内でページ番号を表示するコマンドを呼んでいるため
\thispagestyle{empty} 

\section{節タイトル}

\subsection{参考文献}

参考文献の例を日本語文献\cite{sampleJP},英語文献\cite{sampleEN},日本語URL\cite{sampleURLJP},英語URL\cite{sampleURLEN}に示す.

\subsection{図}

\begin{figure}[h]
	\centering
	\includegraphics[width=3cm]{./img/iwai.png}
	\caption{図のサンプル}
	\label{fig:test}
\end{figure}
\vspace{5mm}

図の例を図\ref{fig:test}に示す.

\subsection{表}

\begin{table}[htb]
	\caption{表のサンプル}
	\begin{center}
	\begin{tabular}{|c|c|} \hline
			A & 1 \\ \hline
			B & 2 \\ \hline
	\end{tabular}
	\end{center}
	\label{table:test}
\end{table}

表の例を表\ref{table:test}に示す.



\subsection{更新}
2025/01/24
\begin{itemize}
    \item 岩井先生を准教授→教授に変更
    \vspace{-1em}
    \item itemizeについてstyファイルの内容(95〜112行目)をコメントアウト(overleafでエラーが出る)
\end{itemize}
予稿は4ページ\\
予稿集にまとめるためページ番号はいらない(ユニぱ要確認)
overleafではcompilerをLatexに変更する\\
latexmkrcは用意しているので表示されるはず

\subsection*{謝辞}

\bibliographystyle{junsrt}
\bibliography{master_yokou}
 
\end{document}
